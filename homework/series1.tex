\documentclass[a4paper,10pt]{article}
\usepackage{xstring}
\usepackage{amssymb} 
\usepackage{amsmath,bbm}
\usepackage[shortlabels]{enumitem} 
\usepackage{private} 
\newcommand{\RN}[1]{%
  \textup{\uppercase\expandafter{\romannumeral#1}}%
}
%opening
\title{Exercises to Statistical Learning WS 2017/18,\\
Series 1 }
\author{\MyDetails}

\begin{document}

\maketitle
\setcounter{section}{1}
\subsection{}
\label{sub:one}

Simplify the following expressions (from Gnedenko)!
\begin{enumerate}[(a)]
  \item $(A+B)(B+C)$ by commutativity:   $=(B+A)(B+C)$\\ simple application of the distributive law yields: $(A+B)(B+C)=B+(AC)$
  \item $(A+B)(A+\overline{B}) \stackrel{distributivity}{=}A+ \underbrace{B\overline{B}}_{=\emptyset}=A$
  \item $(A+B)(\overline{A}+B)(A+\overline{B}) = (B+A)(B+\overline{A}(A\overline{B})=(B+A\overline{A})(A \overline{B}= B(A+\overline{B})= BA +B\overline{B}=AB$
\end{enumerate}
%it is generally a good idea to draw venn diagrams to visualize the problem.
\subsection{}
\label{sub:two}
determine the probabilities of getting the following results when tossing a coin 7 times.
\begin{enumerate}[(a)]
  \item ZWZZWWZ, this sequence is uniquely so it is the probability of tossing 7 times the right thing, hence $\frac{1}{2^7}$
  \item the probability of tossing 4 heads is given by $ \binom{7}{4}\left( \frac{1}{2} \right)^7$ which results in $ \frac{35}{128}$
  \item the probability of tossing at least four times head, is the same probability as that of tossing at least four tails, therefore it is $ \frac{1}{2}$
\end{enumerate}
\subsection{Bayes' formula}
\label{sub:three}
Lets first establish some identities:\\
From the formula for conditional probability
\begin{equation}
  P(A_i|B)=\frac{P(A_iB)}{P(B)}
\end{equation}
we solve for $P(A_iB)$
\begin{equation}
  P(BA_i)=P(A_i|B)P(B)=P(A_iB)=P(B|A_i)P(A_i)
\end{equation}
further for
\begin{equation*}
  \bigcup A_i=\Omega  
\end{equation*}
 and
 \begin{equation*}
 A_i\cap A_j= \emptyset,\quad \forall i\neq j  
 \end{equation*}
 we can expand $P(B)$ to $P(\Omega |B)P(B)$, since $P(\Omega |X)=1$ for $X\neq \emptyset$ ie:
\begin{equation}
  P(B)= \underbrace{  \sum_{j}P(A_j|B)}_{=1} P(B) \stackrel{(2)}{= }  \sum_{j} P(B|A_j)P(A_j)
\end{equation}
it is left to substitute the respective terms  of (2) and (3)  in equation (1) 
\begin{equation}
  P(A_i|B)= \frac{P(B|A_i)P(A_j)}{\sum_{i} P(B|A_j)P(A_j)}
\end{equation}

\subsection{}
\label{sub:sdf}
We have the sensitivity $P(T^+|D^+)=0.98$ specificity $P(T^-|D^-)=0.98$ and the prevalence $P(D^+)=0.01$ given.\\
It is further straightforward to calculate the opposites $P(D^-)=1-P(D^+)=0.99$ and $P(T^+|D^-)=1-P(T^⁻|D^-)=0.02$
\begin{enumerate}[(a)]
  \item the PPV is 
    \begin{equation}
      P(D^+|T^+)= \frac{P(T^+|D^+)P(D^+)}{P(T^+|D^+)P(D^+)+P(T^+|D^-)P(D^-)}
    \end{equation}
    \begin{equation*}
      = \frac{0.98\times 0.01}{0.98\times 0.01+ 0.02\times 0.99}\approx 0.33
    \end{equation*}
  \item we have to take the PPV of the first test as the Prevalence of the second test, since of all persons with a positive first test 0.33\% have a sick fetus, thus 

   \begin{equation*}
     P(D^+|(T^+)^2)= \frac{0.98\times 0.33 }{0.98\times 0.33 +0.2\times 0.67}=0.96
   \end{equation*}
   of course this only works if the tests are completely uncorrelatet.
\end{enumerate}

\subsection{Variance of the binomial distribution}
\label{sub:variance_of_the_binomial_distribution}
recapitulate:\\
The probability function of the binomial distriibution is given by :
\begin{equation*}
  p_n(k)=\binom{n}{k}p^k(1-p)^{n-k}
\end{equation*}
and the  expectation value is  $E(X)=\sum k \binom{n}{k}p^k(1-p)^{n-k}=np$\\
For the Variance we are looking for the expectation value of
\begin{equation*}
  (X-E(X))^2= (k-np)^2= k^2-2knp +n^2p^2
\end{equation*}
The  Variance is found as:
\begin{equation}
  \begin{split}
  V(X)&= \sum_{k} (k^2-2knp+n^2p^2) \binom{n}{k}p^k(1-p)^{n-k}\\
  &= \sum k^2 \binom{n}{k}p^k(1-p)^{n-k} + 2np\underbrace{   \sum k\binom{n}{k}p^k(1-p)^{n-k}}_{= E(X)=np}+ n^2p^2\underbrace{ \sum \binom{n}{k}p^k(1-p)^{n-k}}_{=1}\\
  &=  n^2p^2 -2n^2p^2 +  \underbrace{\sum k \binom{n}{k}p^k(1-p)^{n-k}}_{=E(X)=np} +\sum k(k-1)\binom{n}{k}p^{k}(1-p)^{n-k}\\
  &=  - n^2p^2 + np + n(n-1)p^2 \sum \binom{n-2}{k-2}p^{k-2}(1-p)^{n-k}\\
  &\textrm{change of variables } k'=k-2, \quad n'=n-2\\
  &=  - n^2p^2 +np + n(n-1)p^2 \underbrace{\sum \binom{n'}{k'}p^{k'}(1-p)^{n-k}}_{=1}\\
  &= np-np^2 \\
  &=np(1-p)
  \end{split}
\end{equation}
\subsection{Variance of the Normal distribution}
\label{sub:variance_of_the_normal_distribution}
The probability density of the normal distribution is given by 
\begin{equation*}
  \frac{1}{\sigma \sqrt{2\pi}} \exp(-\frac{(x-a)^2}{2\sigma ^2})
\end{equation*}
the expectation value is $E(X)=a$\\
The variance is:
\begin{equation}
  V(X)\frac{1}{\sigma \sqrt{2\pi}}\int_{-\infty}^{\infty}(x-a)^2 \exp(-\frac{(x-a)^2}{2\sigma ^2})
\end{equation}
we substitute
    $y=\frac{x-a}{\sigma }$    thus:
    \begin{equation}
      x=\sigma y+a , \quad dx=\sigma dy
    \end{equation}

\begin{equation}
V(X)=\frac{\sigma }{ \sqrt{2\pi}}   \int_{-\infty}^{\infty}y^2 e^{-\frac{y^2}{2}}\sigma dy
\end{equation}
This can be solved by partial integration ie by using the fact that:
\begin{equation*}
  uv'= (uv)' -u'v
\end{equation*}
we identify $ y e^{-\frac{y^2}{2}}$ as being more easily integrable, so we write:
\begin{equation*}
  \int_{-\infty}^{\infty} y\left( ye^{-\frac{y^2}{2}} \right)dy=\underbrace{\left[ y( -e^{-\frac{y^2}{2}})\right]_{-\infty}^{\infty}}_{\textrm{odd thus =0 }} \underbrace{-\int_{-\infty}^{\infty}-e^{-\frac{y^2}{2}}dy}_{=\sqrt{2\pi}} 
\end{equation*}
Thus the variance of the normal function is simply $\sigma^2 $ which isn't just a fancy coincidence.








\end{document}
